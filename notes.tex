\documentclass[10pt]{article}

% Core math & graphics
\usepackage{amsmath}
\usepackage{graphicx}
\usepackage[dvipsnames]{xcolor} % load once, with options

% Floats, tables, layout
\usepackage{wrapfig}
\usepackage{tabularx}
\usepackage{array}
\usepackage{booktabs}

% Units/symbol helpers (see notes about gensymb below)
\usepackage{gensymb} % OK; if conflicts with \micro or \perthousand, consider siunitx

% Captions/subcaptions (caption options applied before subcaption)
\usepackage[labelfont=bf]{caption}
\usepackage{subcaption}

% Misc utilities
\usepackage[export]{adjustbox}
\usepackage{setspace}
\usepackage{listings} % load once
\usepackage{mdframed}
\usepackage{tikz}
\usepackage{url}
\usepackage{marginnote}
\usepackage{csquotes}
\usepackage[normalem]{ulem} % keep \emph as italics
\usepackage{multicol}
\usepackage{amsthm}
\usepackage{mathtools}
\usepackage{titlesec}
\usepackage{amssymb}

\titleformat{\section}
  {\color{blue}\normalfont\Large\bfseries}
  {\thesection}{1em}{}

\titleformat{\subsection}
  {\color{violet}\normalfont\large\bfseries}
  {\thesubsection}{1em}{}

% Custom commands
\newcommand{\pd}[2]{\frac{\partial #1}{\partial #2}}
\newcommand{\ub}[2]{\underbracket{#1}_{#2}}
\newcommand{\bb}[1]{\left[ #1 \right]}
\newcommand{\bp}[1]{\left( #1 \right)}
\newcommand{\bcb}[1]{\left\{ #1 \right\}}

% Number equations by section
\numberwithin{equation}{section}

% Theorem definition style
\newtheoremstyle{defwithitnote}      % name
  {3pt}{3pt}                         % space above/below
  {\normalfont}                      % body font
  {}                                 % indent
  {\bfseries}                        % head font (Definition)
  {:}                                % punctuation after head
  {0.5em}                            % space after head
  {\thmname{#1}\thmnumber{ #2}\thmnote{ \textit{(#3)}}}
  %                                 % #1 = name, #2 = number, #3 = optional note
\theoremstyle{defwithitnote}
\newtheorem{definition}{Definition}[section] % number within section

% Nomenclature
\usepackage[intoc]{nomencl}
\makenomenclature

% Hyperlinks (keep near the end)
\usepackage{hyperref}
\hypersetup{
  colorlinks,
  citecolor=black,
  filecolor=black,
  linkcolor=black,
  urlcolor=black
}




\begin{document}
\begin{flushleft}



\onehalfspacing
\tableofcontents
\newpage


\section{Thermodynamics}

\subsection{Definitions}

\begin{definition}[Moles]
A mole is the amount of mass of a substance equal to its molecular weight, i.e., if you have $O_2$, than 32 kg of $O_2$ constitutes one kg-mole and 32 grams would constitute 32 gm-mole.
\end{definition}

\begin{definition}[Avagadro's constant]
Avogradro's constant is the number of particles in one mole of that substance. It is denoted as '$\text{N}_\text{A}$' and is equal to $6.02 \cdot 10^{26}$ particles per kg-mole.    
\end{definition}

\begin{definition}[Universal gas constant]
The universal gas constant is denoted by $\hat{R}$ and is equal to 8.314 J/mol-K. Its units are `energy per amount of substance per temperature increment` as it is a proportionality constant that relates energy to temperature and amount of substance. 
\end{definition}

\begin{definition}[Boltzmann constant]
The Boltzmann constant is denoted by $k_B$ and is equal to $1.381 \cdot 10^{-23}$ J/k. It is the equivalent of the universal gas constant but on a \textit{per particle} basis. 
\end{definition}

\begin{definition}[Dalton's law of partial pressures]
The pressure of a system can be found my summing up the partial pressures of each individual species:
\begin{equation}
    p = \sum_j p_j
\end{equation}
\end{definition}

\begin{definition}[Mass fractions]
The mass fraction of species $j$ ($Y_j$) is defined as the ratio of the density of species $j$ to the density of the mixture:
\begin{equation}
    Y_j = \frac{\rho_j}{\rho}
\end{equation}
\end{definition}

\begin{definition}[Molar fractions]
The molar fraction of species $j$ ($X_j$) can be taken as the ratio of the partial pressure of species $j$ to the pressure of the mixture, or the ratio of the number of moles of species $j$ to the total number of moles in the system:
\begin{equation}
    \frac{p_j}{p} = \frac{\mathcal{N}_j}{\mathcal{N}} = X_j
\end{equation}
\end{definition}

\begin{definition}[Mixture moleular weight]
The mixture molecular weight, $\mathcal{M}$ can be defined as:
\begin{equation}
    \mathcal{M} = \frac{1}{\sum_j Y_j / \mathcal{M}_j} = \sum_j X_j \mathcal{M}_j
\end{equation}
Where $\mathcal{M}_j$ is the molecular weight of species $j$.
\end{definition}

\begin{definition}[Molar fractions to mass fractions] 
The mass fraction of species $j$ can be obtained from the molar fraction by using:
\begin{equation}
    Y_j = X_j \frac{\mathcal{M}_j}{\mathcal{M}}
\end{equation}
The sum of the mass fractions
\end{definition}

\begin{definition}[Mixture gas constant]
The gas constant of the mixture can be found with:
\begin{equation}
    R = \sum_j Y_j R_j
\end{equation}

where $R$ is the mixture specific gas constant ($\hat{R}/\mathcal{M}$) and $R_j$ is the specific gas constant of species $j$.
\end{definition}

\begin{definition}[Calorically perfect gas]
A calorically perfect gas is a gas where the specific heats $c_p$ and $c_v$ are taken to be constant. The enthalpy is then $h = c_p T$, and the internal energy is $e = c_v T$.
\end{definition}

\begin{definition}{Thermally perfect gas}
A thermally perfect gas is one there $c_p$ and $c_v$ are variables are are functions of $T$ only. Differential changes in enthalpy and internal energy is then d$h = c_p$d$T$, and d$e$ = $c_v$d$T$
\end{definition}


\subsection{Equations}

\subsubsection{Equation of state}

\begin{align}
    PV &= mRT  
    \\ pv &= RT
    \\ pV &= \mathcal{N} \hat{R} T
    \\ p &= \rho R T
    \\ p V &= \mathcal{N} \hat{R} T
    \\ p \mathcal{V} &= \hat{R} T
    \\ p &= C\hat{R}T
    \\ pv &= \eta \mathcal{R} T
    \\ pV &= NkT
    \\ p &= nkT
\end{align}

Where

\begin{itemize}
    \item $P$ is the pressure
    \item $V$ is the volume
    \item $v$ is the specific gas constant
    \item $m$ is the mass
    \item $T$ is the temperature
    \item $\rho$ is the density
    \item $\mathcal{V}$ is the molar volume (volume per mole)
    \item $C$ is the concentration of moles per unit volume.
    \item $\eta$ is the mole-mass ratio
    \item $n$ is the number density, or number of particles per unit volume.
\end{itemize}

\section{Statistical thermodynamics}

\begin{definition}[Bose-Einstein (BE) statistics]
    For atoms and molecults made up of an \textit{even} number of elementary particles such as electrons, protons, and neutrons, there is no limit to the number particles that can be in any energy state at a given instant.
\end{definition}

\begin{definition}[Fermi-Dirac (FD) statistics]
    For atoms and molecules made up of an \textit{odd} number of elementary particles, no energy state can be occupied by more than one particle.
\end{definition}

\begin{definition}[$W(N_j)$]

    For these statistics, the number of microstates in a given macrostate is given by:

    \begin{subequations}
        \begin{align}
            W(N_j) &= \prod_j \frac{(N_j + C_j - 1)!}{(C_j - 1)! N_j!} \label{Wbose}
            \\ W(N_j) &= \prod_j \frac{C_j!}{(C_j - N_j)! N_j!} \label{Wfermi}
        \end{align}
        \label{W}
    \end{subequations}

    Here, Eq. [\ref{Wbose}] is used for BE statistics while Eg. [\ref{Wfermi}] is used for FD statistics. $C_j$ is the number of energy states, and $N_j$ is the number of particles in a group.
    
\end{definition}

\begin{definition}[$W_\text{max}$]

    $W_\text{max}$ is defined as the largest of the individual terms in Equations [\ref{W}] and is sometimes refered to as the \textit{most probable microstate}. By solving for this value by setting $\delta (\ln W) = 0$, we can find the particular values of $N_j$ that maximize $W$, giving:

    \begin{equation}
        N_j^* = \frac{C_j}{e^{\alpha + \beta \epsilon_j} \pm 1} \label{Nstar}
    \end{equation}

    Where the minus sign is for BE, and the plus for FD. $\alpha$ and $\beta$ are Lagrange multiplers used to enforce the constraints:

    \begin{equation}
        \sum_j \delta N_j = 0, \hspace{5mm} \sum_j \epsilon_j = \delta N_j
    \end{equation}    
\end{definition}

\begin{definition}[Boltzmann limit]
    The Boltzmann limit is used for conditions where the number of energy states far out numbers the number of group of particles. Therefore, $C_j >> N_j$, and Eqn. [\ref{Nstar}] reduced to:

    \begin{equation}
        N_j^* = C_j e^{-\alpha - \beta \epsilon_j}
    \end{equation}
\end{definition}

\begin{definition}[Boltzmann distribution]
    The Boltzmann distribution, after solving for $\alpha$ and $\beta$ in the above, gives:

    \begin{equation}
        N_j^* = N \frac{C_j e^{-\epsilon_j / kT}}{\sum_j C_j e^{-\epsilon_j / kT}} \label{BoltzmannDist}
    \end{equation}

\end{definition}

\begin{definition}[Molecular partition function]
    The molecular partition function, $Q$, is defined as the denominator in Eq. [\ref{BoltzmannDist}]:

    \begin{equation}
        Q = \sum_j C_j e^{-\epsilon_j / kT}
    \end{equation}
    
    However, this is in terms of groups of energy states having the same energy level $\epsilon_j$. If we we transition to the use of individual energy states, $\epsilon_j$ becomes $\epsilon_i$, and the $C_j$ term is neglected. The equation becomes:
    
    \begin{equation}
        Q = \sum_i e^{-\epsilon_i / kT}
    \end{equation}
\end{definition}

\begin{definition}[Population of energy states]
    The population of energy state $\epsilon_i$ then becomes:
    \begin{equation}
        N_i^{*'} = N \frac{ e^{-\epsilon_i / kT}}{\sum_i e^{-\epsilon_i / kT}}
    \end{equation}    
\end{definition}

\begin{definition}[Energy levels and degeneracy]
    Energy levels are sometimes more convenient to use. An energy level can have multiple energy states in it. If a level contains more than one state, it is said to be \textit{degnerate}, and the number of states inside an energy level $\epsilon_l$ is given as $g_l$. Therefore, the population of particles in energy level $\epsilon_l$, and the partition function is given as:

    \begin{align}
        Q &= \sum_l g_l e^{-\epsilon_l / kT}
        \\ N_l^{*'} &= N \frac{ g_l e^{-\epsilon_l / kT}}{\sum_i g_l e^{-\epsilon_i / kT}}
    \end{align}
\end{definition}

\begin{definition}[Boltzmann's relation]
    Boltzmann's relation finds the relationship between the number of microstates $\Omega$ and the entropy of a system as:
    \begin{equation}
        S = k \ln \Omega
    \end{equation}
\end{definition}

\begin{definition}[Entropy]

    Entropy can then be shown to be of the form:

    \begin{equation}
        S = Nk \left( \ln \frac{Q}{N} + 1 \right) + \frac{E}{T}
    \end{equation}
    
\end{definition}


\begin{definition}[Helmholtz free energy]
    The Helmholtz free energy, $F$, is defined as:
    \begin{equation}
        F = E - TS \label{Helmholtz}
    \end{equation}

    Here, $E$ is the internal energy, $T$ is the temperature, and $S$ the entropy of the system. Differentiation of [\ref{Helmholtz}] yields:

    \begin{equation}
        dF = dE - TdS - SdT
    \end{equation}

    Since $dE - TdS = \mu dN - p dV$, ($\mu$ is the chemical potental per molecule, N is the number of particles, and $p$, $V$ is the pressure and volume of the system), this equation can be rewritten as:

    \begin{equation}
        dF = -SdT - pdV + \mu dN
    \end{equation}

    Which yields the following results:

    \begin{equation}
        S = - \left( \frac{\partial F}{\partial T} \right)_{V, N} \hspace{5mm} p = - \left( \frac{\partial F}{\partial V} \right)_{T, N} \hspace{5mm} \mu = \left( \frac{\partial F}{\partial N} \right)_{T, V}
    \end{equation}
    
\end{definition}

\begin{definition}[Thermodynamic properties of a system]
    The thermodynamic properties of a system, assuming $Q$ and $\epsilon_i$'s are independant of $N$, are:

    \begin{subequations}
        \begin{align}
            F &= -NkT \left(\ln \frac{Q}{N} + 1 \right)
            \\ S &= Nk \left[ \ln \frac{Q}{N} + 1 + T \frac{\partial(\ln Q)}{\partial T} \right]
            \\ E &= NkT^2 \frac{\partial (\ln Q)}{\partial T}
            \\ p &= NkT \frac{\partial (\ln Q)}{\partial V}
            \\ \mu &= -kT \ln \frac{Q}{T}
        \end{align}
    \end{subequations}
\end{definition}

\begin{definition}[Total partion function and energy]

    As will be defined shortly, the total partion function is the product of all energy-type specific partion functions, i.e.:

    \begin{equation}
        Q = Q_{tr} Q_{rot} Q_{vib} Q_{el} = Q_{tr} \cdot \prod_{int} Q_{int}
    \end{equation}

    The total internal energy of a molecule is:

    \begin{equation}
        \epsilon = \epsilon_{tr} + \epsilon_{rot} + \epsilon_{vib} + \epsilon_{el}
    \end{equation}

    This also works for specific heat, $c_v$.
    
\end{definition}

\begin{definition}[Partition function for translation energy]
The partition function for translation energy is:

\begin{equation}
    Q_{tr} = V \left( \frac{2 \pi m k T}{h^2} \right)^\frac{3}{2}
\end{equation}

The translational internal energy can be found as:

\begin{equation}
    E_{tr} = \frac{3}{2} NkT \hspace{5mm} e_{tr} = \frac{3}{2}R T
\end{equation}

Where the first equation is per N molecules, and the second equation is on a per-unit mass basis. The specific heat is then:

\begin{equation}
    c_{v_{tr}} = \left(\frac{\partial e_{tr}}{\partial T} \right)_{v} = \frac{3}{2} R
\end{equation}
    
\end{definition}

\subsection{Electronic energies}

Since monoatomic molecules do not have rotational or vibrational energies, the only internal energy is the electronic one. However, this still applies to diatomic molecules.

\begin{definition}[Electronic internal energy]
    The electronic internal energy can largely be ignored at moderate temperatures for some species of gases. However, they can become important at higher temperatures. Because the spacing between the levels can be so large, usually only a few degenerate levels are accounted for. Electronic energy accounts for when electrons populate higher electronic energy levels.
\end{definition}

\begin{definition}[Electronic partition function]

    The electronic partition function is:

    \begin{equation}
        Q_{el} = \sum_l g_l e^{-\epsilon_l / kT} = \sum_l g_l e^{\theta_l / T}
    \end{equation}

    Where $\theta_l = \epsilon_l / k$ are the \textit{characteristic temperatures for electronic excitation}. Energy of the ground level is taken to be 0, so the first term is just $g_0$ with no exponential. If we assume that just the first two terms are used, the internal energy becomes:
    
    \begin{equation}
        e_{el} = R \theta_1 \frac{(g_1 / g_0)e^{-\theta_l / T}}{1 + (g_1/g_0)e^{-\theta_1 / T}}
    \end{equation}

    and the specific heat:

    \begin{equation}
        c_{v_{el}} = R \left( \frac{\theta_1}{T} \right)^2 \frac{(g_1/g_0) e^{-\theta_1 / T}}{[1 + (g_1/g_0) e^{-\theta_1 / T}]^2}
    \end{equation}
    
\end{definition}

\subsection{Diatomic internal energies}


\newpage
\section{High Temperature Transport Models}

\textit{References for this section}

\begin{enumerate}
    \item \textit{A Review of Reaction Rates and Thermodynamic and Transport Properties for an 11-Species Air Model for Chemical and Thermal Nonequilibrium Calculations to 30,000 K}, Gupta, Yos, Thompson, Lee, 1990 NASA RP 1232.
    \item \textit{An assessment of Transport Property Methodologies for Hypersonic Flows}, Palmer, 1971, AIAA.
\end{enumerate}


\subsection{Collision integral definitions}

\begin{definition}[Collision cross sections]
    The collision cross section has infinitely many values. Four of them are used for fluid dynamics to determine transport properties. The collision cross sections are, in general, the weighted averages of the cross sections for collisions between species $i$ and $j$ (sometimes denoted $s$ and $r$). They are defined as:
    
    \begin{equation}
        \pi \bar{\Omega}_{ij}^{(l,s)} = \frac{\int_{0}^{\infty} \int_{0}^{\pi} \exp(-\gamma^2) \gamma^{2s+3}(1 - \cos^l \chi) 4 \pi \sigma_{ij} \sin\chi d \chi d\gamma}{\int_{0}^{\infty} \int_{0}^{\pi} \exp(-\gamma^2) \gamma^{2s + 3} (1 - \cos^l{\chi}) \sin{\chi} d\chi d \gamma }
    \end{equation}

    Variables are:

    \begin{itemize}
        \item $\sigma_{ij} = \sigma_{ij}(\chi, g)$ is the differential-scattering cross section for the pair ($i,j$).
        \item $\chi$ is the scattering angle in the center of mass system.
        \item $g$ is the relative velocity of the colliding particles.
        \item $\gamma$ is the reduced velocity equal to $\displaystyle \bp{ \bb{\frac{m_i m_j}{2(m_i + m_j)kT}}^\frac{1}{2} }g$
    \end{itemize}

    In the collision integral symbol, $l$ controls how strong the contribution of the differential-scattering angle ($\chi$) is. $s$ is a weighting on the reduced velocity $(\gamma)$ of the particles. Of course, the subscript $(ii)$ refers to the collision integral for similar particles. Here is a list of where certain collision integrals show up:

    \begin{itemize}
        \item Viscosity ($\mu$) is a function of $\pi \bar{\Omega}_{ij}^{(2,2)}$ only.
        \item Translational component of thermal conductivity ($\kappa_\text{tr}$) is a function of $\pi \bar{\Omega}_{ij}^{(2,2)}$ only.
        \item The internal component of thermal conductivity ($\kappa_\text{int}$) is a function of $\pi \bar{\Omega}_{ij}^{(1,1)}$ only or a function of $\pi \bar{\Omega}_{ij}^{(1,1)}$ and $\pi \bar{\Omega}_{ij}^{(2,2)}$, depending on the expression used.
        \item The diffusion coefficient $D_{ij}$ is a function of $\pi \bar{\Omega}_{ij}^{(1,1)}$ only.
    \end{itemize}
\end{definition}


\begin{definition}[Gupta's curve fits for collision integrals]
    Gupta utilizes curve fits for the collision integrals. They have the following form:

    \begin{equation}
        \pi \bar{\Omega}_{ij}^{(l,s)} = \bb{\exp{D}} T^{\bb{A (\ln T)^2 + B \ln T + C}}
    \end{equation}

    \begin{equation}
        B_{i,j}^{*(2,2)} = \bb{\exp{C}} T^{\bb{A (\ln T)^2 + B \ln T}}
    \end{equation}


\end{definition}


\begin{definition}[Chapman-Enskog procedure]
    
\end{definition}

\subsection{Transport properties of a single species}

\begin{definition}[Viscosity of a single species]

    The viscosity of a gas containing a single species is:  

    \begin{equation}
        \mu_i = \frac{5}{16} \frac{\sqrt{\pi m_i k T}}{\pi \bar{\Omega}_{ii}^{(2,2)}} = 8.3861 \cdot 10^{-6} \frac{\sqrt{\mathcal{M}_i T}}{\pi \bar{\Omega}_{ij}^{(2,2)}} \hspace{5mm} \text{units = } \frac{\text{N - s}}{\text{m}^2}
    \end{equation}

    Where $\mathcal{M}_i$ is the molecular weight of species $i$, and $T$ is the temperature in Kelvin. 

\end{definition}

\begin{definition}[Thermal conductivity of a single species]
    
    
    For the thermal conductivity of a single species, we have a number of contributions. The total $\kappa$ of the species is the sum of its internal energy specific contributions, i.e.:
    
    \begin{equation}
        \kappa = \kappa_\text{tr} + \kappa_\text{rot} + \kappa_\text{vib} + \kappa_\text{el} = \kappa_\text{tr} + \kappa_\text{int}
    \end{equation}

    Where the internal energies that are not translational have been grouped into $\kappa_\text{int}$. 

    \begin{equation}
        \kappa_{\text{tr}, i} = \frac{5}{2} \frac{\mu_i}{\mathcal{M}_i} (c_{v,i})_\text{tr} = \frac{15}{4} \frac{\hat{R}}{\mathcal{M}_i}\mu_i \hspace{5mm} \text{units = } \frac{\text{J}}{\text{m-s-K}}
    \end{equation}

    \begin{equation}
        \kappa_{\text{int}, i} = \frac{6}{5}\frac{c_{p_{\text{int},i}}}{\hat{R}} \frac{\hat{R}}{\mathcal{M}_i} \mu_i \frac{\pi \bar{\Omega}_{ii}^{(2,2)}}{\pi \bar{\Omega}_{ii}^{(1,1)}}
    \end{equation}

    Here, $c_{p_{\text{int},i}}$ is the total specific heat with the translation specific heat subtracted.

\end{definition}


\subsection{Transport properties of a mixture}

For species summation and Wilke's mixing rule, $mu_j$ and $\kappa_j$ are calculated using the above equations for single species viscosity and conductivity.


\begin{definition}[Species summation]

    The concept of the species summation version of finding mixture transport properties is quite simple. It assumes no interaction between species and is expressed as:

    \begin{equation}
        \mu = \sum_{j=1}^\text{NS} X_j \mu_j \hspace{7.5mm} \kappa = \sum_{j=1}^\text{NS} X_j \kappa_j
    \end{equation}
    
\end{definition}



\begin{definition}[Wilke's mixing rule]

    Both viscosity and thermal conductivity can be approximated using Wilke's mixing rule. It uses a quantity, $\phi_i$, which is expressed as:

    \begin{equation}
        \phi_i = \sum_{j=1}^\text{NS} \frac{X_j \bb{1 + \sqrt{\frac{\mu_i}{\mu_j}} \bp{\frac{\mathcal{M}_j}{\mathcal{M}_i}}^\frac{1}{4}}^2}{\sqrt{8 \bp{1 + \frac{\mathcal{M}_i}{\mathcal{M}_j}}}} = \sum_{j=1}^\text{NS} \frac{X_j \bb{1 + \sqrt{\frac{\pi \bar{\Omega}_{jj}^{(2,2)}}{\pi \bar{\Omega}_{ii}^{(2,2)}}}}^2}{\sqrt{8 \bp{1 + \frac{\mathcal{M}_i}{\mathcal{M}_j}}}}
    \end{equation}

    Then the viscosity and thermal conductivity is found as:

    \begin{equation}
        \mu = \sum_{j=1}^\text{NS} \frac{X_j \mu_j}{\phi_j} \hspace{5mm} \kappa = \sum_{j=1}^\text{NS} \frac{X_j \kappa_j}{\phi_j}
    \end{equation}    
\end{definition}


\begin{definition}[Chapman-Enskog approximation method]
    For obtaining mixture transport quantities using Chapman-Enskog approximations, a number of new variables are used. This is the most rigorous method for finding the properties. The units of $\Delta$ is in centimeters-second, while $\alpha_{ij}$ is unitless.

    \begin{equation}
        \Delta_{ij}^{(1)} = \frac{8}{3} (1.5460 \cdot 10^{-20}) \bb{\frac{2 \mathcal{M}_i \mathcal{M}_j}{\pi N_A k T (\mathcal{M}_i + \mathcal{M}_j)}}^\frac{1}{2} \pi \bar{\Omega}_{ij}^{(1,1)}
    \end{equation}

    \begin{equation}
        \Delta_{ij}^{(2)} = \frac{16}{5} (1.5460 \cdot 10^{-20}) \bb{\frac{2 \mathcal{M}_i \mathcal{M}_j}{\pi N_A k T (\mathcal{M}_i + \mathcal{M}_j)}}^\frac{1}{2}  \pi \bar{\Omega}_{ij}^{(2,2)}
    \end{equation}

    For electron-electron collisions, we have:

    \begin{equation}
        \pi \bar{\Omega}_{ee}^{(2,2)} = 1.29 Q_e \cdot 10^{16}
    \end{equation}

    For ion-ion collisions, we have:

    \begin{equation}
        \pi \bar{\Omega}_{II}^{(2,2)} = 1.36 Z^4 Q_e \cdot 10^{16}
    \end{equation}

    Then for both:

    \begin{equation}
        \pi \bar{\Omega}_{ee}^{(1,1)} = \pi \bar{\Omega}_{II}^{(1,1)} = 0.795 Z^4 Q_e \cdot 10^{16}
    \end{equation}

    Here, $Z=1$ for singly ionized species (for aerospace applications this is fine), and:

    \begin{equation}
        Q_e = \frac{e^4}{(kT)^2} \ln \Lambda \hspace{5mm} \text{cm}^2
    \end{equation}

    \begin{align}
        \Lambda &= \bb{\frac{9 (kT)^3}{4 \pi e^6 n_e} + \frac{16 (kT)^2}{e^4 n_e^{2/3}}}^{1/2}
        \\ &= \bb{2.09 \cdot 10^{-2} \bp{\frac{T^4}{10^{12} p_e}} + 1.52 \bp{\frac{T^4}{10^{12}p_e}}^{2/3}}^{1/2}
    \end{align}

    Here, $p_e$ is the electron pressure in atmospheres, $n_e$ is the number density of electrons, and $e = 4.8 \cdot 10^{-10}$  esu. These equations are only useful unter for $p_e < $ some limiting value, which should be true in aerospace applications:

    \begin{equation}
        p_{em} = 0.0975 \bp{\frac{T}{10^3}}^4
    \end{equation}

    Depending on the data given, the Coulomb cross section may need to be altered. Gupta and Yos give collision cross sections at the above limiting electron pressure, so at any other pressure the cross section needs to be changed:

    \begin{align}
        \frac{\pi \bar{\Omega}_{ij}^{(l,s)}(p_e)}{\pi \bar{\Omega}_{ij}^{(l,s)}(p_{em})} &= \ln \Lambda (p_e)
        \\ &= \frac{1}{2} \ln \bb{2.09 \cdot 10^{-2} \bp{\frac{T}{1000 p_e^{1/4}}}^4 + 1.52 \bp{\frac{T}{1000 p_e^{1/4}}}^{8/3}}
    \end{align}

    Finally, $\alpha_{ij}$ is needed for the heavy-particle translational component of thermal conductivity:

    \begin{equation}
        \alpha_{ij} = 1 + \frac{[1 - (\mathcal{M}_i / \mathcal{M}_j)][0.45 - 2.54(\mathcal{M}_i / \mathcal{M}_j)]}{[1 + (\mathcal{M}_i / \mathcal{M}_j)]^2}
    \end{equation}

    Assuming thermochemical nonequilibrium, the mixture viscosity \textit{appears} to not be a function of vibrational or rotation energy, and is thus found with:

    \begin{equation}
        \mu = \sum_{i=1}^{\text{NS} - 1} \bb{\frac{\frac{\mathcal{M}_i}{N_A}X_i}{\sum_{j=1}^\text{NS - 1} X_j \Delta_{ij}^{(2)}(T) + X_e \Delta_{ie}^{(2)}(T_e) }} + \frac{\frac{\mathcal{M}_e}{N_A} X_e }{\sum_{j=1}^\text{NS} X_j \Delta_{ej}^{(2)}(T_e)}
    \end{equation}

    Where $X_j$ is the molar fraction, $T$ is the Temperature, and $T_e$ is the electron temperature. In constrast to viscosity, the thermal conductivity is dependant on all energy modes, and therefore has contributions from each mode. They are:
    \vspace{5mm}

    Translational component without electron-heavy-particle and electron-electron collision:

    \begin{equation}
        K_\text{tr}^* = (2.3901 \cdot 10^{-8}) \frac{15}{4} k \sum_{i=1}^\text{NS-1} \bb{\frac{X_i}{\sum_{j=1} \alpha_{ij} \Delta_{ij}^{(2)}(T) + 3.54 X_e \Delta_{ie}^{(2)}(T_e) }}
    \end{equation}

    Rotational component. "p.e." means  "partial excitation" while "f.e." mean "full excitation". If it is in equilibrium with translational temperature, $T_\text{rot} = T$:

    \begin{equation}
        (K_\text{rot})_\text{p.e.} = (2.3901 \cdot 10^{-8})k \sum_\text{i=mol.} \bb{\frac{\bb{\frac{c_{p,i}(T_\text{rot})}{\hat{R}} - \frac{5}{2}} X_i}{\sum_{j=1}^\text{NS-1}X_j \Delta_{ij}^{(1)}(T) + X_e \Delta_{ie}^{(1)}(T_e)}}
    \end{equation}

    \begin{equation}
        (K_\text{rot})_\text{f.e.} = (2.3901 \cdot 10^{-8})k \sum_\text{i=mol.} \bb{\frac{X_i}{\sum_{j=1}^\text{NS-1}X_j \Delta_{ij}^{(1)}(T) + X_e \Delta_{ie}^{(1)}(T_e)}}
    \end{equation}


    Vibrational component:

    \begin{equation}
        (K_\text{vib})_\text{p.e.} = (2.3901 \cdot 10^{-8})k \sum_\text{i=mol.} \bb{\frac{\bb{\frac{c_{p,i}(T_\text{vib})}{\hat{R}} - \frac{7}{2}} X_i}{\sum_{j=1}^\text{NS-1}X_j \Delta_{ij}^{(1)}(T) + X_e \Delta_{ie}^{(1)}(T_e)}}
    \end{equation}

    \begin{equation}
        (K_\text{vib})_\text{f.e.} = (2.3901 \cdot 10^{-8})k \sum_\text{i=mol.} \bb{\frac{X_i}{\sum_{j=1}^\text{NS-1}X_j \Delta_{ij}^{(1)}(T) + X_e \Delta_{ie}^{(1)}(T_e)}}
    \end{equation}

    Electronic component:

    \begin{equation}
        K_\text{el} = (2.3901 \cdot 10^{-8})k \sum_\text{i=1}^\text{NS-1} \bb{\frac{\bb{\frac{c_{p,i}(T_\text{el})}{\hat{R}} - \frac{9}{2}} X_i}{\sum_{j=1}^\text{NS-1}X_j \Delta_{ij}^{(1)}(T) + X_e \Delta_{ie}^{(1)}(T_e)}}
    \end{equation}

    Free electron component:

    \begin{equation}
        K_\text{e} = (2.3901 \cdot 10^{-8})k \frac{X_e}{\sum_{j=1}^\text{NS-1}X_j \Delta_{ej}^{(2)}(T_e) + X_e \Delta_{ee}^{(2)}(T_e)}
    \end{equation}




\end{definition}


\newpage
\section{Chemical Rate Kinetics}

\subsection{Rate equations}

\begin{definition}[Reactions]
    For multicomponent gas with NS reacting species and NR different reactions, the stoichiometric relations for the change in reactants and products are:

    \begin{equation}
        \sum_{i=1}^\text{NS} \nu_i' X_i \leftrightarrows \sum_{i=1}^\text{NS} \nu_i'' X_i, 
        \hspace{5mm} r=1, \dots , NR
    \end{equation}

    $\nu_i'$ and $\nu_i''$ are the stoichiometric coefficients for reactants and products respectively. $k_{f,r}$ governs the forward reaction (right arrow), and $k_{b,r}$ governs the backward reaction (left arrow). 
    
\end{definition}

\begin{definition}[Mass production]
    The net rate of MASS production of a species per unit volume has a number of different expressions. Consider the reaction:

    \begin{equation}
        O_2 + M \longrightarrow 2O + M
        \label{ReactionExample1}
    \end{equation} 

    Here, M is an arbitrary collision partner, deemed a \textit{third-body}. Concentration of species $j$ is the moles of $j$ per unit volume. $[O_2]$ is the concentration of $O_2$. The time rate of change of the concentration of $O$ is then:

    \begin{equation}
        \frac{d [O]}{dt} = 2k [O_2][M]
    \end{equation}

    Where $k$ is the reaction rate. This equation gives the rate at which Eq. [\ref{ReactionExample1}] goes from left to right. Therefore, $k$ is actually the \textit{forward} reaction rate, $k_f$. The two previous equations should then be:

    \begin{equation}    
        O_2 + M \overset{k_f}{\longrightarrow} 2O + M
    \end{equation} 

    \begin{equation}
        \frac{d [O]}{dt} = 2k_f [O_2][M]
    \end{equation}

    Of course, the right to left reaction is the \textit{backwards} reaction and is governed by the backwards rate coefficient:

    \begin{equation}
        O_2 + M \underset{k_b}{\longleftarrow} 2O + M
    \end{equation}

    \begin{equation}
        \frac{d [O]}{dt} = -2 k_b [O]^2[M]
    \end{equation}

    You must combine these forward and backwards equations because both reactions happen. This gives:

    \begin{equation}
        O_2 + M \overset{k_f}{\underset{k_b}{\longleftrightarrow}} 2O + M
    \end{equation}

    \begin{equation}
        \frac{d [O]}{dt} = 2 k_f [O_2][M] -2 k_f [O]^2[M]
    \end{equation}

    The net rate equation for an arbitrary reaction is then:

    \begin{equation}
        \frac{d [X_i]}{dt} = (\nu_i'' - \nu_i') \left \{ k_f \prod_i [X_i]^{\nu_i'} - k_b \prod_i [X_i]^{\nu_i''} \right \}
    \end{equation}

    Where $\nu'$ is the stoichiometric for a reactant species, and $\nu''$ is for product species.

\end{definition}

\begin{definition}[Reaction rate relations]

    The equilibrium constant is very useful to find one reaction rate from the other. There are two equilibrium constants: $K_c$ is the equilibrium constant based on concentrations, and $K_p$ is the equilibrium constant based on partial pressures. They are related through:

    \begin{equation}
        K_c = \bp{\frac{1}{\hat{R}T}}^{\sum_i \nu_i}K_p
    \end{equation}

    Where it is raised to a sum due to the complicated nature of the equilibrium constant. Typically, the rate coefficient has variable units, since it depends on the number of molecules in the reaction, which gives rise to a key concept:
    
    \vspace{5mm} 
    \textit{For most chemical kinetic models, the reaction's reactants always consist of 2 species, while the reaction's products may contain two or three species. Therefore, the forwards rate coefficients $k_f$ should always have the same units. Since the equilibrium constant is used to find $k_b$, no unit conversions are necessary besides perhaps from cgs to mks units.}
    
    \vspace{5mm}
    Curve fits of the following form are used to find the equilibrium constant:

    \begin{align*}
        (K_\text{eq,r}) &= \exp(\beta)
        \\ \beta &= A_1 \frac{T}{10^4} + A_2 + A_3 \ln \frac{10^4}{T} + A_4 \frac{10^4}{T} + A_5 \bp{\frac{10^4}{T}}^2  
    \end{align*}

    With the equilibrium constant of a reaction known, you can find the forward/backward reaction rates as stated above with:

    \begin{equation}
        \frac{k_{f,r}(T)}{k_{b,r}(T)} = K_\text{eq,r}
    \end{equation}

\end{definition}

\begin{definition}[Arrhenius equations]

    Both forward and backward rate coefficients can be found using the Arrhenius or modified Arrhenius equations, though usually the forward rate is computed and the backward rate is found with the equilibrium constant. The Arrhenius equation has the form:

    \begin{equation}
        k_{f,r} = C e^{-E_a/kT}
    \end{equation}

    Where C is a constant, $E_a$ is the activation energy, $k$ is the Boltzmann constant, and $T$ is the temperature. The modified Arrhenius form (more widely used) is:

    \begin{equation}
        k_{f,r} = C T^\eta e^{-E_0/kT}
    \end{equation}

    Here, $\eta$ is a reaction specific exponent, and $E_0$ depends on the reaction (could be a dissociation energy, could be 0 for recombination, etc.)
    
\end{definition}

\subsection{Reaction Types}

\begin{definition}[Dissociation Reaction]

    A dissociation reaction occurs when a molucule collides with some other species and one or more atoms "detatch" from the molecule. Examples are:

    \begin{align}
        O_2 + M &\leftrightarrows 2O + M
        \\ N_2 + M &\leftrightarrows 2N + M
        \\ NO + M &\leftrightarrows N + O + M
        \\ O_2 + N &\leftrightarrows NO + O 
        \label{shuffle}
        \\ N_2 + O &\leftrightarrows NO + N
        \\ N_2 + O_2 &\leftrightarrows 2NO
        \\ N + O &\leftrightarrows NO^+ + e^- 
        \label{dissrecomb}
    \end{align}

    Eq. [\ref{shuffle}] is an example of a \textit{bimolecular exchange} reaction. Eq. [\ref{dissrecomb}] is called a \textit{dissociative-recombination}  reaction since the recombination of the $NO^+$ ion with an electron produces a dissociated product.

    \vspace{5mm} To be continued...
    
\end{definition}

\begin{definition}[Efficiencies]
    For certain reactions like dissociation, the outcome of the reaction is the same regardless of the other reactant. These are called \textit{third-body reactions}, since the secondary reactant does not change chemically. The third-body is usually represented with an $M$ as we have already discussed:

    \begin{equation}
        N_2 + M \rightleftarrows N + N + M \label{thirdbody}
    \end{equation}

    This $M$ can be any species, and the Arrhenius equation is nearly identical for any $M$, with only a slight change in the leading ceofficient $C$. This coefficient change is called an \textit{efficiency} (denoted here as $\epsilon$), and the algebra will be shown as to how this is beneficial to use. A list of efficiencies in certain reactions are shown in Table \ref{efficiencies}.

    \begin{table}[!h]
    \centering
    \caption{Change in modified Ahhrenius constant $C$ for some third-bodies in $N_2$ dissociation.}
    \begin{tabular}{p{25mm} p{15mm} p{15mm}}
    \hline
    Third body & C & $\epsilon$ \\
    \hline
    $M = O_2$ & 7.0e21 & 1.0 \\
    $M = NO$ & 7.0e21 & 1.0 \\
    $M = O$ & 3.0e22 & 4.2857 \\
    \hline
    \end{tabular}
    \label{efficiencies}
    \end{table}

    Since $7.0e21$ is the smallest value, this is given an efficiency of 1.0, and then any other reaction's efficiency is given as some multiplier of this base value. The physical meaning of efficiencies is how "efficient" that third body is at making that reaction happen. $\epsilon$ is larger for atoms, since they are more efficient in sending their energy into dissociation of a molecule. We can now do some algebra on the Arrhenius equation by letting $C$ = $\epsilon C_c$, then $\gamma$ = $C_c T^\eta e^{E_0/kT}$. The forward  and backwards reaction constants are then:

    \begin{equation}
        k_f = \epsilon \gamma, \hspace{5mm} k_b = \frac{\epsilon \gamma}{K_{eq}}
    \end{equation}

    For a third body reaction, the change in the concentration of $N_2$ due to a single third-body $M$ is then:

    \begin{align*}
        \frac{d[N_2]}{dt} &= (0 - 1)\left\{ \epsilon \gamma [N_2][M] - \frac{\epsilon \gamma}{K_{eq}}[N]^2[M] \right\} 
        \\ &= -1 \cdot \epsilon [M] \underbracket{ \left\{ \gamma [N_2] - \frac{\gamma}{K_{eq}}[N]^2 \right\}}_{R}
    \end{align*}

    $R$ is independant of the third-body, so it is a constant and only needs to be computed once. Summing over contributions from all independant reactions gives:

    \begin{align*}
        \frac{d[N_2]}{dt} &= - \bb{\sum_{i} \epsilon_i [M_i]} R
    \end{align*}

    Where:

    \begin{equation}
        \sum_i \epsilon_i [M_i] = \epsilon_{N_2} [N_2] + \epsilon_{O_2} [O_2] + \epsilon_{NO} [NO] + \epsilon_{N} [N] + \epsilon_{O} [O] + \dots
    \end{equation}

    $R$ is retained for the production of $N$ atoms as well, you only need to change the coefficient from the $\nu_i'' - \nu_i''$ term:

    \begin{align*}
        \frac{d[N]}{dt} = \sum_i \epsilon_i [M_i] \ub{(2 - 0)}{\nu_{N}'' - \nu_N'} R
    \end{align*}

\end{definition}


\subsection{Example rate equation solution}

    Here, and example is done to find the rate of production of $NO$. Consider the following reactions that all contain $NO$:

    \begin{subequations}
        \begin{align}
            NO + O_2 &\leftrightarrows N + O + O_2
            \\ NO + N_2 &\leftrightarrows N + O + N_2
            \\ NO + NO &\leftrightarrows N + O + NO
            \\ NO + O &\leftrightarrows N + O + O
            \\ NO + N &\leftrightarrows N + O + N
            \\ O_2 + N &\leftrightarrows NO + O
            \\ N_2 + O &\leftrightarrows NO + N
        \end{align}
    \end{subequations}

    The forward and backward rate coefficients for the reactions are $k_{f,a}$ and $k_{b,a}$ for the first reaction, all the way down with consistent lettering with the equation number. The entire equation is then:
    
    \begin{align*}
        \frac{d[NO]}{dt} &= -\bcb{\gamma [NO] - \frac{\gamma}{K_{eq}} [N][O]} \sum_i \epsilon_i [M]
        \\ &+ \bcb{k_{f,f} [O_2][N] - k_{b,f} [NO][O]}
        \\ &+ \bcb{k_{f,g} [N_2][O] - k_{b,g} [NO][N] }
    \end{align*}
    

\newpage
\section{Internal energy relaxation}

\subsection{Translational-Vibrational relaxation}

The source term for vibrational energy due to transfer of energy between translational and vibrational energy modes from the Landau-Teller model is:

\begin{equation}
    Q_{t-v}^{\text{LT}} = \sum_s Q_{t-v,s}^\text{LT}
\end{equation}

Where 

\begin{equation}
    Q_{t-v,s}^\text{LT} = \rho_s \frac{e_{vs}^* - e_{vs}}{<\tau_s>}
\end{equation}

$e_{vs}^*$ is the equilibrium vibrational energy, which is evaluated at the tran-rotational tempertature, $T_{tr}$, while $e_{vs}$ is the actual vibrational energy evaluated with $T_v$. The molar-averaged relaxation time, $<\tau_s>$ is defined by:

\begin{equation}    
    <\tau_s> = \frac{\sum_r X_r}{\sum_r X_r / \tau_{sr}}
\end{equation}

This sum happens for all particles $r$, and molecules $s$. The term $\tau_sr$ is the relaxation time between species and has two parts - the Millikan-White part, as well as the high-temperature Park correction. The Millikan-White part uses the equations:

\begin{equation}
    \tau_{sr}^\text{MW} = \frac{101325}{p} \exp{[A_{sr} (T^{-1/3} - B_{sr}) - 18.42]}
\end{equation}

Where $A_sr$ and $B_{sr}$ are either set constants or computed from:

\begin{align}
    \mu_{sr} &= \frac{\mathcal{M}_s \mathcal{M}_r}{\mathcal{M}_s + \mathcal{M}_r}
    \\ A_{sr} &= 1.16 \times 10^{-3} \mu_{sr}^{1/2} \theta_{vs}^{4/3}
    \\ B_{sr} &= 0.015 \mu_{sr}^{1/4}
\end{align}

The Park correction is:

\begin{align}
    \tau_{sr}^\text{P} &= \frac{1}{\bar{c} \sigma n}
    \\ \bar{c} &= \sqrt{\frac{8 \hat{R} T}{\pi}}
    \\ \sigma &=  3 \times 10^{-21} \bp{\frac{50000}{T}}^2
    \\ n &= \frac{p}{k_b T}
\end{align}




\subsection{Chemical-Vibrational relaxation}




\newpage
\section{Thermochemical Nonequilibrium CFD}


\subsection{Species conservation equation}

    \begin{equation}
    \underbracket{ \pd{}{t} \rho_s}_1 + \underbracket{\pd{}{x_j} \rho_s u_j}_2 = \underbracket{\pd{}{x_j} (\rho D_s \pd{}{x_j} y_s)}_3 + \underbracket{\dot{\omega}_s}_4
    \end{equation}

    The terms are:

    \begin{enumerate}
        \item Rate of change of mass of species $s$.
        \item Flux of mass due to convection.
        \item Molecular diffusion flux of mass.
        \item Chemical production rate.
    \end{enumerate}

    The variables are:
    \begin{itemize}
        \item $\rho_s$ is the density of species s
        \item $u_j$ is the mass-averaged velocity in the $j$-direction
        \item $D_s$ is the effective diffusion coefficient of species s
        \item $y_s$ is the molar fraction of species s
        \item $\dot{\omega}_s$ is the chemical production rate of species s
    \end{itemize}

\subsection{Total continuity equation}
    For chemically reacting flows, the total continuity equation is unaffected. It is:

    \begin{equation}
        \pd{\rho}{t} + \nabla \cdot (\rho \textbf{u}) = 0
    \end{equation}

    Since no mass can be created or destroyed, the condition of atomic conservation is applied, which yields:

    \begin{equation}
        \sum_{s} \dot{w}_s = 0
    \end{equation}

\subsection{Mixture momentum conservation}

    \begin{equation}
        \underbracket{\pd{}{t} \rho u_i}_{1} + \underbracket{\pd{}{x_j} \rho u_i u_j}_{2} = \underbracket{-\pd{p}{x_i}}_{3} + \underbracket{\pd{}{x_j} \left[ \mu \left( \pd{u_i}{x_j} + \pd{u_j}{x_i} \right) - \frac{2}{3} \mu \pd{u_k}{x_k} \delta_{ij} \right]}_4
    \end{equation}

    The terms are:

    \begin{enumerate}
        \item Rate of change of $i$th component of momentum.
        \item Flux of $i$th component of momentum due to convection.
        \item Pressure forces in $i$th direction.
        \item Viscous terms acting in $i$th direction.
    \end{enumerate}

    Here, the mixture pressure is defined with Dalton's law of partial pressures, and the pressure of an individual heavy species is:

    \begin{equation}
        p_s = \frac{\rho_s \hat{R}T}{\mathcal{M}_s}
    \end{equation}

    For electron pressure, we get:

    \begin{equation}
        p_e = \frac{\rho_e \hat{R} T_e}{\mathcal{M}_e}
    \end{equation}

    Where the $e$ subscript denotes an electron property.

\subsection{Vibrational energy conservation}

    \begin{multline}
        \ub{\pd{}{t} \rho e_v}{1} + \ub{\pd{}{x_j}\rho e_v u_j}{2} = \ub{\pd{}{x_j} \bp{\eta_v \pd{T_v}{x_j}}}{3} + \ub{\pd{}{x_j} \bp{\rho \sum_{s=1}^\text{NS} h_{v,s} D_s \pd{y_s}{x_j}}}{4} \\ + \ub{\sum_\text{s = mol.} \rho_s \frac{(e_{v,s}^* - e_{v,s})}{<\tau_s>}}{5} + \ub{\sum_\text{s=mol.} \rho_s \frac{(e_{v,s}^{**} - e_{v,s})}{<\tau_{es}>}}{6} + \ub{\sum_\text{s=mol.}\dot{w}_s \hat{D}_s}{7}
    \end{multline}

    The terms are:

    \begin{enumerate}
        \item Rate of change of vibrational energy per unit volume.
        \item The flux of vibrational energy due to convection.
        \item The conduction of vibrational energy due to vibrational temperature gradients.
        \item The diffusion of vibrational energy due to molecular concentration gradients.
        \item The energy exchange (relaxation) between vibrational and translational modes due to collisions within the cell.
        \item The energy exchange between vibrational and electronic modes.
        \item The vibrational lost/gained due to molecular depletion (dissociation) or production (recombination).
    \end{enumerate}

    The variables are:

    \begin{itemize}
        \item $e_v$ is the vibrational energy per unit mass, defined as $\displaystyle e_v = \sum_{s=1}^\text{NS} \frac{\rho_s e_{v,s}}{\rho}$. $e_v$ is defined as a function of $T_v$, the vibrational temperature.
        \item $\eta_v$ is the vibrational thermal conductivity.
        \item $h_{v,s}$ is the vibrational enthalpy. It is identical to $e_{v,s}$.
        \item $e_{v,s}^*$ is the vibrational energy of species $s$ at the translational-rotational temperature.
        \item $e_{v,s}^{**}$ is the vibrational energy at the electron temperature. 
        \item $<\tau_s>$ is the characteristic relaxation times for translational-vibrational (T-V) energy exchange.
        \item $<\tau_{es}>$ is the characteristic relation times for electron-vibrational (e-V) energy exchange.
        \item $\hat{D}_s$ is the vibration energy level representative of those molecules of species $s$ which are created or destroyed (recombined or dissociated) because of their high vibrational quantum numbers.
    \end{itemize}

\subsection{Electron and electronic excitation energy conservation}

    \begin{multline}
        \ub{\pd{}{t} \rho e_e}{1} + \ub{\pd{}{x_j} \bb{u_j(\rho e_e + p_e)}}{2} = \ub{u_j \pd{p_e}{x_j} }{3} + \ub{\pd{}{x_j} \bp{\eta_e \pd{T_e}{x_j}}}{4} + \ub{\pd{}{x_j} \bp{\rho \sum_{s=1}^\text{NS} h_{e,s} D_s \pd{y_s}{x_j}}}{5} \\ + \ub{2 \rho_e \frac{3}{2} \hat{R}(T - T_e) \sum_{s=1}^\text{NS - 1} \frac{\nu_{es}}{\mathcal{M}_s}}{6}  - \ub{\sum_\text{s=ion}\dot{n}_{e,s} \hat{I}_s}{7} - \ub{\sum_\text{s=mol.} \rho_s \frac{e_{v,s}^{**} - e_{v,s}}{<\tau_{es}>}}{8} - \ub{Q_\text{rad}}{9}
    \end{multline}

    Terms are:

    \begin{enumerate}
        \item Rate of change of electronic energy per unit volume.
        \item The flux of electronic enthalpy due to convection.
        \item The word done on electrons by an electron field by an electron pressure gradient.
        \item The conduction of electronic energy due to electron temperature gradient.
        \item The diffuction of electronic energy due to concentration gradients.
        \item The energy exchange due to elastic collisions between electrons and heavy particles.
        \item Energy loss due to electron impact ionization (note the sum indices sum over ionized species only).
        \item The relaxation due to inelastic collisions between electrons and molecules. 
        \item Rate of energy loss due to radiation caused by electronic transitions.
    \end{enumerate}

    The variables are:
    \begin{itemize}
        \item $e_e$ is the electronic energy per unit mass and contains contributions from electronic energy levels. It is defined as $\displaystyle e_e = \sum_{s=1}^\text{NS} \frac{\rho_s e_{e,s}}{\rho}$
        \item  $e_{e,s}$ is the electronic energy per unit mass of species $s$ and is defined as a function of $T_e$.
        \item $T_e$ is the electron temperature.
        \item $\eta_e$ is the is the electronic thermal conductivity.
        \item $h_{e,s}$ is the electronic enthalpy per unit mass and is idential to $e_{e,s}$ for all species except free electrons. For free electrons it is defined as: $ \displaystyle h_{e,e} = e_{e,e} + \frac{\hat{R}T_e}{\mathcal{M}_e}$
        \item $\nu_{e,s}$ is the effective collision frequency of electrons with heavy particles.
        \item $\dot{n}_{e,s}$ is the molar rate of production of species $s$ by electron impact ionizations.
        \item $\hat{I}_s$ is the energy per unit mole lost by a free electron in producing species $s$ through electron impact ionization.
        \item $Q_\text{rad}$ is the radiant energy transfer rate due to electron transitions.
    \end{itemize}

\subsection{Total energy conservation}

    \begin{multline}
        \ub{\pd{}{t} \rho E}{1} + \ub{\pd{}{x_j} \rho H u_j}{2} = \ub{\pd{}{x_j} \bp{\eta \pd{T}{x_j} + \eta_v \pd{T_v}{x_j} + \eta_e \pd{T_e}{x_j}}}{3} \\ + \ub{\pd{}{x_j} \bp{\rho \sum_{s=1}^\text{NS} h_s D_s \pd{y_s}{x_j}}}{4} + \ub{\pd{}{x_j} \bb{u_i \mu \bp{\pd{u_i}{x_j} + \pd{u_j}{x_i}}}}{5} - \ub{Q_\text{rad}}{6}
    \end{multline}
    
    The terms are:

    \begin{enumerate}
        \item Rate of change of total energy per unit volume.
        \item The flux of total enthalpy due to convection.
        \item The conduction of energy due to temperature, vibrational temperature, and electron temperature gradients.
        \item The diffusion of energy due to concentration gradients.
        \item The word done by shear forces.
        \item The rate of energy loss due to radiation caused by electronic transitions.
    \end{enumerate}

    The variables are:

    \begin{itemize}
        \item $E$ is the total energy per unit volume, defined as $\displaystyle E = \frac{u_i u_i}{2} + \sum_{s=1}^\text{NS} \frac{\rho_s e_s}{\rho}$
        \item $H$ is the total enthalpy per unit mass, defined as $\displaystyle H = E + \frac{p}{\rho}$
        \item $\eta$ is the frozen thermal conductivity of heavy particles. The part in which exchanges of translational and rotational energy occur.
    \end{itemize}

\subsection{Two temperature model}

    Everything before this assumes a three-temperature model, where there is a single temperature dedicated to translational-rotational energy, vibrational energy, and electronic-electron energies. The two temperature model assumes that one temperature can be used for describing the energy of translational-rotational energies, and then the rest (vib-electronic-electron) are desibed by the second temperature, $T_V$. Therefore:
    
    \begin{equation}
        T_v = T_e = T_V
    \end{equation}

    By combining the vibrational and electronic energy equations, we get the vibrational-electronic energy conservation equation:

    \begin{multline}
        \ub{ \pd{}{t} \rho e_V}{1} + \ub{\pd{}{x_j} \rho e_V u_j}{2} = \ub{- p_e \pd{u_j}{x_j}}{3} + \ub{\pd{}{x_j} \bb{(\eta_v + \eta_e) \pd{T_V}{x_j}}}{4} \\ + \ub{\pd{}{x_j} \bp{\rho \sum_{s=1}^\text{NS} h_{V,s} D_s \pd{y_s}{x_j}}}{5} + \ub{\sum_{s=mol.} \rho_s \frac{(e_{v,s}^* - e_{v,s})}{<\tau_s>}}{6} \\ + \ub{2 \rho_e \frac{3}{2} \hat{R}(T - T_V) \sum_{s=1}^\text{NS - 1} \frac{\nu_{es}}{\mathcal{M}_s}}{7}  - \ub{\sum_\text{s=ion} \dot{n}_{e,s} \hat{I}_s}{8} + \ub{ \sum_{s=mol.} \dot{w}_s \hat{D}_s}{9}- \ub{Q_\text{rad}}{10}
    \end{multline}

    Where:

    \begin{align}
        e_V &= e_v + e_e
        \\ h_{V,s} &= h_{v,s} h_{e,s}
    \end{align}


\subsection{Derivatives}

\subsubsection{Partial derivatives of pressure ($P$)}

\begin{subequations}
    \begin{align}
        \pd{p}{\rho_s} &= \bp{\frac{\hat{R}}{\mathcal{M}_s} - c_{vs} \frac{R}{c_v}} T + \frac{R}{c_v} \bb{\frac{1}{2} (u^2 + v^2 + w^2) - h_s^\circ}
    \\ \pd{p}{(\rho u)} &= -u \frac{R}{c_v}
    \\ \pd{p}{(\rho v)} &= -v \frac{R}{c_v}
    \\ \pd{p}{(\rho w)} &= -w \frac{R}{c_v}
    \\ \pd{p}{E_v} &= -\frac{R}{c_v}
    \\ \pd{p}{E} &= \frac{R}{c_v}
    \end{align}
\end{subequations}

\subsubsection{Partial derivatives of total energy ($E$)}

\begin{subequations}
    \begin{align}
        \pd{E}{\rho_s} &= \bp{c_{vs} - \frac{c_v}{R} \frac{\hat{R}}{\mathcal{M}_s}}T + \frac{1}{2} (u^2 + v^2 + w^2) + e_v + h_s^\circ
        \\ \pd{E}{u} &= \rho u 
        \\ \pd{E}{v} &= \rho v
        \\ \pd{E}{w} &= \rho w
        \\ \pd{E}{e_v} &= \sum_{s=1}^\text{NS} \rho_s
        \\ \pd{E}{p} &= \frac{c_v}{R}
    \end{align}
\end{subequations}


\subsubsection{Partial derivatives of $T$}

\begin{subequations}
    \begin{align}
        \pd{T}{\rho_s} &= \frac{1}{\rho c_v} \bp{-c_{vs}T + \frac{1}{2} (u^2 + v^2 + w^2) - h_s^\circ}
        \\ \pd{T}{(\rho u)} &= -\frac{u}{\rho c_v}
        \\ \pd{T}{(\rho v)} &= -\frac{v}{\rho c_v}
        \\ \pd{T}{(\rho w)} &= -\frac{w}{\rho c_v}
        \\ \pd{T}{E_v} &= -\frac{1}{\rho c_v}
        \\ \pd{T}{E} &= \frac{1}{\rho c_v}
        \\ \pd{c_r}{\rho_s} &= (\delta_{rs} - c_r) \frac{1}{\rho}
    \end{align}
\end{subequations}

\subsubsection{Partial derivatives of vibrational temperature ($T_v$)}

\begin{subequations}
    \begin{align}
    \pd{T_v}{\rho_i} &= -e_{vi} / \sum_{s=1}^\text{NS} \bb{\rho_s \bp{\pd{e_{vs}}{T_v}}}
    \\ \pd{T_v}{(\rho u)} &= 0
    \\ \pd{T_v}{(\rho v)} &= 0
    \\ \pd{T_v}{(\rho w)} &= 0
    \\ \pd{T_v}{E_v} &= 1 / \sum_{s=1}^\text{NS} \bb{\rho_s \bp{\pd{e_{vs}}{T_v}}}
    \\ \pd{T_v}{E} &= 0
    \end{align}
\end{subequations}

\subsubsection{Partial derivatives of vibrational energy ($e_vs$)}

\begin{subequations}
    \begin{align}
        \pd{e_{vs}}{\rho_i} = \bp{\pd{e_{vs}}{T_v}}_{T_v} \pd{T_v}{\rho_i}, &\hspace{10mm} \pd{e_{vs}^*}{\rho_i} =  \bp{\pd{e_{vs}}{T_v}}_{T_v  = T} \pd{T_v}{\rho_i}
        \\ \pd{e_{vs}}{(\rho u)} = \bp{\pd{e_{vs}}{T_v}}_{T_v} \pd{T_v}{(\rho u)}, &\hspace{10mm} \pd{e_{vs}^*}{(\rho u)} = \bp{\pd{e_{vs}}{T_v}}_{T_v  = T} \pd{T_v}{(\rho u)}
        \\ \pd{e_{vs}}{(\rho v)} = \bp{\pd{e_{vs}}{T_v}}_{T_v} \pd{T_v}{(\rho v)}, &\hspace{10mm} \pd{e_{vs}^*}{(\rho v)} = \bp{\pd{e_{vs}}{T_v}}_{T_v  = T} \pd{T_v}{(\rho v)}
        \\ \pd{e_{vs}}{(\rho w)} = \bp{\pd{e_{vs}}{T_v}}_{T_v} \pd{T_v}{(\rho w)}, &\hspace{10mm} \pd{e_{vs}^*}{(\rho w)} = \bp{\pd{e_{vs}}{T_v}}_{T_v  = T} \pd{T_v}{(\rho w)}
        \\ \pd{e_{vs}}{E_v} = \bp{\pd{e_{vs}}{T_v}}_{T_v} \pd{T_v}{E_v}, &\hspace{10mm} \pd{e_{vs}^*}{E_v} = \bp{\pd{e_{vs}}{T_v}}_{T_v  = T} \pd{T_v}{E_v}
        \\ \pd{e_{vs}}{E} = \bp{\pd{e_{vs}}{T_v}}_{T_v} \pd{T_v}{E}, &\hspace{10mm} \pd{e_{vs}^*}{E} = \bp{\pd{e_{vs}}{T_v}}_{T_v  = T} \pd{T_v}{E}
    \end{align}    
\end{subequations}



\subsection{Jacobians}

\subsubsection{Flux Jacobians}

\subsubsection{Source term Jacobians - Third body reactions}

As stated, for third body reactions, the Arrhenius rate coefficient $k_f$ can be expressed as an efficiency times a base rate coefficient that is independant of the third body:

\begin{equation}
    k_f(T) = \epsilon_i \gamma(T) 
\end{equation}

where 

\begin{equation}
    \gamma(T) = C_c T^\eta e^{E_a/T}
\end{equation}

For third-body reactions, the rate of production of species $s$ can be expressed as:

\begin{equation}
    \frac{d [X_i]}{dt} = (\nu_i'' - \nu_i') \underbracket{\sum_{i=1}^\text{NS} \epsilon_i [M_i]}_{E_\text{eff}} \underbracket{\left \{ \gamma(T_f) \prod_i [X_i]^{\nu_i'} - \frac{\gamma(T_b)}{K_c(T_b)} \prod_i [X_i]^{\nu_i''} \right \}}_{L_r}
\end{equation}

Here, $T_f$ is the temperature used to compute the forward rate coefficient, and $T_b$ is the temperature used to compute the backward rate coefficient. For now, we are only interest in the two underbracketed sections, $E_\text{eff}$ and $L_r$. The reaction rate (without the stoichiometric coefficients in the front) can then be described by:

\begin{equation}
    \mathcal{R}_r = E_\text{eff} L_r
\end{equation}

The partial derivative of this reaction rate with respect to some conserved variable $U_j$ is then:

\begin{equation}
    \pd{\mathcal{R}_r}{U_j} = \pd{E_\text{eff}}{U_j} L_r + E_\text{eff} \pd{L_r}{U_j}
\end{equation}

The two derivatives on the right hand side are then:

\begin{equation}
    \pd{E_{eff}}{U_j} = 
    \begin{cases}
        E_{eff} + \frac{\epsilon_j}{\mathcal{M}_j} (1 - \rho_j), \hspace{5mm} \text{if $U_j$ = $\rho_j$}
        \\ 0, \hspace{5mm} \text{if $U_j$ = $\rho u$, $\rho v$, etc.}
    \end{cases}
\end{equation}

\begin{multline}
    \pd{L_r}{U_j} = \bcb{\pd{T}{U_j} \pd{}{T}[\gamma(T_f)]} \bb{\frac{\rho_{s,f}}{\mathcal{M}_{s,f}}}^{\nu_s'} + \gamma(T_f) \bcb{\pd{}{U_j} \bb{\frac{\rho_{s,b}}{\mathcal{M}_{s,b}}}^{\nu_s'}}
    \\ - \bcb{\pd{T}{U_j} \pd{}{T} \bb{\frac{\gamma(T_b)}{K_c(T_b)}}} \bb{\frac{\rho_{s,b}}{\mathcal{M}_{s,b}}}^{\nu_s''} - \frac{\gamma(T_b)}{K_c(T_b)} \bcb{\pd{}{U_j} \bb{\frac{\rho_{s,b}}{\mathcal{M}_{s,b}}}^{\nu_s''}}
\end{multline}

Of course, in the second equation, more needs to be expanded out. These expansions yield:

\begin{equation}
    \pd{}{T} \gamma(T) = \gamma(T) \bb{\frac{\eta q}{T_{tr}} - \frac{q E_a}{T_{tr}^{1+q} T_v^{1-q}}}
\end{equation}

\begin{equation}
    \pd{}{U_j} \bb{\frac{\rho_{s}}{\mathcal{M}_s}}^{\nu_s^*} = \nu_s^* \bb{\frac{\rho_{s}}{\mathcal{M}_s}}^{\nu_s^* - 1} \cdot \frac{1}{\mathcal{M}_s}
\end{equation}

\begin{equation}
    \pd{}{T} K_{c,m}(T) = \frac{K_{c,m}(T)}{T} \bb{\sum_s(\nu_s'' - \nu_s') \frac{H_s^\circ}{\hat{R}T} + \nu_m}
\end{equation}

Putting it together, the $L_r$ derivative is:

\begin{multline}
    \pd{L_r}{U_j} = \gamma(T_f) \bb{\frac{\rho_{s,f}}{\mathcal{M}_{s,f}}}^{\nu_s'}\times \textcolor{blue}{\bb{\pd{T}{U_j} \bp{\frac{\eta q^f}{T_{tr}} - \frac{q^f E_a}{(T_{tr}^{1+q} T_v^{1-q})_f}} + \underbracket{\frac{\nu_s'}{\rho_{s,f}}}_{\textcolor{black}{\text{if } U_j = \rho_j}} }}
    \\ - \frac{\gamma(T_b)}{K_c(T_b)} \bb{\frac{\rho_{s,b}}{\mathcal{M}_{s,b}}}^{\nu_s''} \times \cdots 
    \\ \textcolor{blue}{\left\{ \pd{T}{U_j} \bb{\frac{\eta q^b}{T_{tr}} - \frac{q^b E_a}{(T_{tr}^{1+q} T_v^{1-q})_b} - \frac{1}{T_b} \bp{\sum_{i=1}^\text{NS} (\nu_i^{''} - \nu_i') \frac{H_s^\circ}{\hat{R}T} + \nu_m} } + \underbracket{\frac{\nu_{s}''}{\rho_{s,b}}}_{\textcolor{black}{\text{if } U_j = \rho_j}} \right\} } 
\end{multline}

\subsubsection{Source term Jacobians - Normal reactions}


For normal reaction, we can revert to the usual notation of $k_f$ and $k_b$ for the forward and backward rate coefficients with no efficiencies. A reaction is then:

\begin{equation}
    R_r = k_f(T_f) \prod_i \bb{\frac{\rho_i}{\mathcal{M}_i}}^{\nu_i'} - \frac{k_f(T_b)}{K_c(T_b)} \prod_i \bb{\frac{\rho_i}{\mathcal{M}_i}}^{\nu_i''}
\end{equation}

The derivative of this reaction rate with respect to some conserved variable $U_j$ is then:

\begin{multline}
    \pd{R_r}{U_j} = k_f(T_f) \bb{\frac{\rho_{s,f}}{\mathcal{M}_{s,f}}}^{\nu_s'} \times \textcolor{blue}{\bb{\pd{T}{U_j} \bp{\frac{\eta q^f}{T_{tr}} - \frac{q^f E_a}{(T_{tr}^{1+q} T_v^{1-q})_f}} + \underbracket{\frac{\nu_s'}{\rho_{s,f}}}_{\textcolor{black}{\text{if } U_j = \rho_j}} }}
    \\ - \frac{k_f(T_b)}{K_c(T_b)} \bb{\frac{\rho_{s,b}}{\mathcal{M}_{s,b}}}^{\nu_s''} \times \cdots 
    \\ \textcolor{blue}{\left\{ \pd{T}{U_j} \bb{\frac{\eta q^b}{T_{tr}} - \frac{q^b E_a}{(T_{tr}^{1+q} T_v^{1-q})_b} - \frac{1}{T_b} \bp{\sum_{i=1}^\text{NS} (\nu_i^{''} - \nu_i') \frac{H_s^\circ}{\hat{R}T} - \nu_m} } + \underbracket{\frac{\nu_{s}''}{\rho_{s,b}}}_{\textcolor{black}{\text{if } U_j = \rho_j}} \right\} }
\end{multline}


\subsubsection{Source term Jacobians - Vibrational relaxation}

Recall the vibrational energy source term:

\begin{equation}
    Q_{t-v,s} = \rho_s \frac{e_{vs} - e_{vs}^*}{<\tau_s>}
\end{equation}

\begin{equation}
    e_{vs} = \frac{\hat{R}}{\mathcal{M}_s} \frac{\theta_{vs}}{\exp{(\theta_{vs} / T_v)} - 1}
\end{equation}

The derivative is then:

\begin{multline}
    \pd{Q_{t-v,s}}{U_j} = \bb{\pd{}{U_j} \rho_s} \frac{e_{vs} - e_{vs}^*}{<\tau_s>} 
    \\ + \rho_s \bcb{ \bb{\pd{}{U_j} (e_{vs} - e_{vs}^*)} \frac{1}{<\tau_s>} + (e_{vs} - e_{vs}^*)  \bp{-\tau_s ^{-2} \bb{\pd{}{U_j} <\tau_s>}}}
\end{multline}

Where:

\begin{equation}
    \pd{e_{vs}}{T_v} = \bp{\frac{\theta_{vs}}{T_v}}^2 \frac{\exp{(\theta_{vs} / T_v)}}{\bb{\exp{(\theta_{vs} / T_v) - 1}}^2}
\end{equation}

And 

\begin{equation}
    \pd{e_{vs}^*}{T_v} = \left. \pd{e_{vs}}{T_v} \right|_{T_v = T_{tr}}
\end{equation}

We now aim to find the partial derivatives of the relaxation time:

\begin{equation}
    \pd{<\tau_s>}{U_j} = \pd{}{U_j} \frac{\sum_{r=1}^\text{NS} X_r}{\sum_{r=1}^\text{NS} X_r / \tau_{sr}}
\end{equation}

This gives:

\begin{eqnarray}
    \pd{<\tau_s>}{U_j} = \underbracket{ \bb{\pd{}{U_j} \sum_{r=1}^\text{NS} X_r} \frac{1}{\sum_{r=1}^\text{NS} X_r / \tau_{sr}}}_{0} + \sum_{r=1}^\text{NS} X_r \bb{\pd{}{U_j} \bp{\sum_{r=1}^\text{NS} \frac{X_r}{\tau_{sr}}}^{-1}}
\end{eqnarray}

Since the first term is zero, this leaves us with:

\begin{equation}
    \pd{<\tau_s>}{U_j} = \bb{\sum_{r=1}^\text{NS} X_r} \bb{\pd{}{U_j} \bp{\sum_{r=1}^\text{NS} \frac{X_r}{\tau_{sr}}}^{-1} }
\end{equation}

Differentiating the second bracketed term:

\begin{equation}
    \pd{}{U_j} \bp{\sum_{r=1}^\text{NS} \frac{X_r}{\tau_{sr}}}^{-1} = - \bp{\sum_{r=1}^\text{NS} \frac{X_r}{\tau_{sr}}}^{-2} \bb{\pd{}{U_j} \sum_{r=1}^\text{NS} \frac{X_r}{\tau_sr} }
\end{equation}

Bringing the derivative inside, using the product rule and splitting the sums gives the final derivative as:

\begin{equation}
    \boxed{\pd{<\tau_s>}{U_j}  = - \bb{\sum_{r=1}^\text{NS} X_r} \bb{\sum_{r=1}^\text{NS} \frac{X_r}{\tau_{sr}}}^{-2} \bcb{ \textcolor{blue}{\sum_{r=1}^\text{NS} \bb{\pd{X_r}{U_j} \frac{1}{\tau_{sr}}}} + \textcolor{red}{\sum_{r=1}^\text{NS} \bb{X_r \bp{\pd{}{U_j} \frac{1}{\tau_{sr}}}}}}}
\end{equation}

The blue term inside expands using an identity:

\begin{align}
    X_r &= \frac{\rho_r / \mathcal{M}_r}{\sum_{i=1}^\text{NS} \rho_i / \mathcal{M}_i}
    \\ \pd{X_r}{\rho_j} &= \bb{\pd{}{\rho_j} \frac{\rho_r}{\mathcal{M}_r}} \frac{1}{\sum_{i=1}^\text{NS} \rho_i / \mathcal{M}_i} + \frac{\rho_r}{\mathcal{M}_r} \bb{\pd{}{\rho_j} \frac{1}{\sum_{i=1}^\text{NS} \rho_i / \mathcal{M}_i}}
\end{align}

Giving:

\begin{eqnarray}
    \pd{X_r}{\rho_j} = \frac{1}{\sum_{i=1}^\text{NS} \rho_i / \mathcal{M}_i} \bb{\frac{\delta_{jr}}{\mathcal{M}_r} - \frac{X_r}{\mathcal{M}_j}}
\end{eqnarray}

Noting that:

\begin{eqnarray}
     \frac{1}{\sum_{i=1}^\text{NS} \rho_i / \mathcal{M}_i} = \frac{\mathcal{M}_\text{mix}}{\rho_\text{mix}}
\end{eqnarray}

These mixture properties will be given no subscript going forward. This gives:

\begin{eqnarray}
    \boxed{\pd{X_r}{\rho_j} = \frac{\mathcal{M}}{\rho} \bb{\frac{\delta_{jr}}{\mathcal{M}_r} - \frac{X_r}{\mathcal{M}_j}}}
\end{eqnarray}

And all other derivatives WRT conserved variables are 0. $\delta_{jr}$ is the Kronecker delta. Then $\rho_j$ = $\rho_r$, it is 1, otherwise it is 0. We still need to sum this with the reciprocal of $\tau_{sr}$. This produces:

\begin{equation}
    \textcolor{blue}{ \sum_{r=1}^\text{NS} \bb{\pd{X_r}{U_j} \frac{1}{\tau_{sr}}}  }
\end{equation}

A possible trick uses the following identity. Consider a constant $C$, and the fact that:

\begin{equation}
    \sum_{r=1}^\text{NS} \pd{X_r}{U_j} = \pd{}{U_j} \underbracket{\sum_{r=1}^\text{NS} X_r}_{1} = 0
\end{equation}

Clearly, multiplying this by $C$ does not change the result:

\begin{equation}
    \sum_{r=1}^\text{NS} \bb{\pd{X_r}{U_j} C} = 0
\end{equation}

If you subtract this from the above blue term, you get:

\begin{equation}
    \textcolor{blue}{\sum_{r=1}^\text{NS} \bb{\pd{X_r}{U_j} \frac{1}{\tau_{sr}}}} - \sum_{r=1}^\text{NS} \bb{\pd{X_r}{U_j} C} = \sum_{r=1}^\text{NS} \bb{\pd{X_r}{U_j} \bp{\frac{1}{\tau_{sr}} - C}}
\end{equation}

If $C$ is a constant that is chosen properly, you may be able to help the stiffness of the problem. To start, choose

\begin{equation}
    C = \sum_{k=1}^\text{NS} \frac{X_k}{\tau_{sk}}
\end{equation}

For a diatomic species $s$, this is essentially a mean relaxation time. Since this does not change when summing over $r$, this will not effect the result of the above equation, but may help with stiffness since we are calculating based off of how far away from the mean the relaxation times are.
Therefore, the derivative of $<\tau_s>$ may be expressed as:

\begin{multline}
    \pd{<\tau_s>}{U_j}  = - \bb{\sum_{r=1}^\text{NS} X_r} \bb{\sum_{r=1}^\text{NS} \frac{X_r}{\tau_{sr}}}^{-2} \times \dots 
    \\ \bcb{ \textcolor{blue}{\sum_{r=1}^\text{NS} \bb{\pd{X_r}{U_j} \bp{\frac{1}{\tau_{sr}} - \sum_{k=1}^\text{NS} \frac{X_k}{\tau_{sk}}}}} + \textcolor{red}{\sum_{r=1}^\text{NS} \bb{X_r \bp{\pd{}{U_j} \frac{1}{\tau_{sr}}}}}}
\end{multline}

The red term expands to the following \textit{without} Park's correction:

\begin{align}
    \sum_{r=1}^\text{NS} \bb{X_r \bp{\pd{}{U_j} \frac{1}{\tau_{sr}}}} &= - \sum_{r=1}^\text{NS} \bb{X_r \tau_{sr}^{-2} \pd{\tau_{sr}}{U_j}}
\end{align}

The Millikan-White relaxation derivative is:

\begin{equation}
    \boxed{\pd{\tau_{sr}^\text{MW}}{U_j} = -\tau_{sr}^\text{MW} \textcolor{blue}{\bb{\frac{101325}{p} \pd{p}{U_j} + \frac{1}{3} \pd{T}{U_j} T^{-4/3} A_{sr}}}}
\end{equation}

So the completed red term is:

\begin{align}
     \sum_{r=1}^\text{NS} \bb{X_r \bp{\pd{}{U_j} \frac{1}{\tau_{sr}}}} &= \sum_{r=1}^\text{NS} \bb{- X_r \tau_{sr}^{-2} \bb{-\tau_{sr} \bp{\frac{101325}{p} \pd{p}{U_j} + \frac{1}{3} \pd{T}{U_j} T^{-4/3} A_{sr}}}}
    \\ &=  \sum_{r=1}^\text{NS} \bb{\frac{X_r}{\tau_{sr}} \bp{\frac{101325}{p} \pd{p}{U_j} + \frac{1}{3} \pd{T}{U_j} T^{-4/3} A_{sr}}}
\end{align}


With Park's correction we have:

\begin{equation}
    \tau_{sr} = \tau_{sr}^\text{MW} + \tau_{sr}^\text{P}
\end{equation}

So:

\begin{equation}
    \pd{\tau_{sr}}{U_j} = \pd{\tau_{sr}^\text{MW}}{U_j} + \pd{\tau_{sr}^\text{P}}{U_j}
\end{equation}

The harder derivative (somehow) is Park's term:

\begin{equation}
    \pd{\tau_{sr}^\text{P}}{U_j} = \pd{}{U_j} \bb{\frac{1}{\bar{c} \sigma n}} = -(\bar{c} \sigma n)^{-2} \bcb{\pd{\bar{c}}{U_j} \sigma n + \bar{c} \bb{\pd{\sigma}{U_j} n + \sigma \pd{n}{U_j}}}
\end{equation}


With:

\begin{equation}
    \bar{c} = \sqrt{\frac{8 \hat{R} T}{\pi \sum_{i=1}^\text{NS} X_i \mathcal{M}_i}}, \hspace{5mm} n = \frac{p}{kT}, \hspace{5mm} \sigma = \sigma_v' \bp{\frac{50000}{T}}^2
\end{equation}

We begin with:

\begin{equation}
    \pd{\bar{c}}{U_j} = \sqrt{\frac{8 \hat{R}}{\pi}} \pd{}{U_j} \bb{T^{1/2} \bp{\sum_{i=1}^\text{NS} X_i \mathcal{M}_i}^{-1/2}}
\end{equation}

\begin{multline}
     \pd{\bar{c}}{U_j} = \sqrt{\frac{8 \hat{R}}{\pi}} \times \left\{ \frac{1}{2} T^{-1/2} \pd{T}{U_j} \bp{\sum_{i=1}^\text{NS} X_i \mathcal{M}_i}^{-1/2} \right.
     \\ \left. - \frac{1}{2} T^{1/2} \bp{\sum_{i=1}^\text{NS} X_i \mathcal{M}_i}^{-3/2} \textcolor{red}{ \bb{\pd{}{U_j} \sum_{i=1}^\text{NS} \bp{X_i \mathcal{M}_i}}} \right\}
\end{multline}

The term in read expands as the following:

\begin{align}
    \bb{\pd{}{U_j} \sum_{i=1}^\text{NS} \bp{X_i M_i}} &= \sum_{i=1}^\text{NS} \bb{\mathcal{M}_i \pd{X_i}{U_j}}
    \\ &= \sum_{i=1}^\text{NS} \bb{\mathcal{M}_i \bp{\frac{\mathcal{M}}{\rho} \bb{\frac{\delta_{ji}}{\mathcal{M}_i} - \frac{X_i}{\mathcal{M}_j}}}}
    \\ &= \frac{\mathcal{M}}{\rho} \bcb{ \sum_{i=1}^\text{NS} [\delta_{ij}] - \sum_{i=1}^\text{NS} \bb{\frac{\mathcal{M}_i X_i}{\mathcal{M}_j}}   }
    \\ &= \frac{\mathcal{M}}{\rho} \bb{1 - \frac{\mathcal{M}}{\mathcal{M}_j}}
\end{align}


The derivative of $\bar{c}$ is then:

\begin{multline}
    \pd{\bar{c}}{U_j} = \sqrt{\frac{8 \hat{R}}{\pi}} \left\{\frac{1}{2}T^{-1/2} \pd{T}{U_j} \bp{\sum_{i=1}^\text{NS} X_i \mathcal{M}_i}^{-1/2} \right. 
    \\ \left. - \frac{1}{2} T^{1/2} \bp{\sum_{i=1}^\text{NS} X_i \mathcal{M}_i}^{-3/2} \bb{\frac{\mathcal{M}}{\rho} \bp{1 - \frac{\mathcal{M}}{\mathcal{M}_j}}}  \right\}
\end{multline}


\begin{equation}
    \boxed{\pd{\bar{c}}{U_j} = \bar{c} \textcolor{blue}{\bb{\frac{1}{2T} \pd{T}{U_j} \underbracket{- \frac{1}{2 \rho} \bp{1 - \frac{\mathcal{M}}{\mathcal{M}_j}}}_{\text{if $U_j$ = $\rho_j$}}}}}
\end{equation}

The derivative of $n$ is far more straight forward:

\begin{align}
    \pd{n}{U_j} &= \pd{p}{U_j} \frac{1}{kT} + \frac{p}{k} \bp{\pd{}{U_j} \frac{1}{T}}
    \\ &= \pd{p}{U_j} \frac{1}{kT} - \frac{p}{k} \frac{1}{T^2} \pd{T}{U_j}
\end{align}

\begin{equation}
    \boxed{\pd{n}{U_j} = n \bb{ \frac{1}{p} \pd{p}{U_j} - \frac{1}{T} \pd{T}{U_j}}}
\end{equation}

Finally, the derivative wrt to $\sigma$:

\begin{align}
    \pd{\sigma}{U_j} &= \pd{}{U_j} \bb{\sigma_v' \bp{\frac{50000}{T}}^2}
    \\ &= \sigma_v' 50000^2 \pd{}{U_j} \frac{1}{T^2}
    \\ &= \sigma_v' 50000^2 \bp{-2 T^{-3} \pd{T}{U_j}}
\end{align}

\begin{equation}
    \boxed{\pd{\sigma}{U_j} = \sigma \bb{-\frac{2}{T} \pd{T}{U_j}}}
\end{equation}


Putting it all together for Park's derivative, we have:

\begin{multline}
    \pd{\tau^\text{P}}{U_j} = - (\bar{c}\sigma n)^{-2} \left\{ \bar{c} \bb{\frac{1}{2 T} \pd{T}{U_j} - \frac{1}{2 \rho} \bp{1 - \frac{\mathcal{M}}{\mathcal{M}_j}}} \sigma n \right.
    \\ \left. + \bar{c} \sigma \bb{-\frac{2}{T} \pd{T}{U_j}}n + \bar{c} \sigma n \bb{\frac{1}{p} \pd{p}{U_j} - \frac{1}{T} \pd{T}{U_j}} \right\}
\end{multline}

\begin{multline}
    \pd{\tau^\text{P}}{U_j} = - (\bar{c} \sigma n) ^{-2} \left\{ \bar{c} \sigma n \left[ \frac{1}{2T} \pd{T}{U_j} - \frac{1}{2 \rho} \bp{1 - \frac{\mathcal{M}}{\mathcal{M}_j}} \right. \right.
    \\ \left. \left. - \frac{2}{T} \pd{T}{U_j} + \frac{1}{p} \pd{p}{U_j} - \frac{1}{T} \pd{T}{U_j} \right] \right\}
\end{multline}

Collecting terms, cancelling out, and flipping signs gives:

\begin{equation}
    \boxed{ \pd{\tau^\text{P}}{U_j} = \tau^\text{P} \textcolor{blue}{ \bb{\frac{5}{2T} \pd{T}{U_j} - \frac{1}{p} \pd{p}{U_j} + \underbracket{\frac{1}{2 \rho} \bp{1 - \frac{\mathcal{M}}{\mathcal{M}_j}}}_{\text{if $U_j$ = $\rho_j$}}}}} 
\end{equation}

If Millikan-White is used with Park's correction, the final derivative for $\tau_{sr}$ is:

\begin{multline}
    \pd{\tau_{sr}}{U_j} = -\tau_{sr}^\text{MW} \textcolor{blue}{\bb{\frac{101325}{p} \pd{p}{U_j} + \frac{1}{3} \pd{T}{U_j} T^{-4/3} A_{sr}}}
    \\ + \tau^\text{P} \textcolor{blue}{ \bb{\frac{5}{2T} \pd{T}{U_j} - \frac{1}{p} \pd{p}{U_j} + \underbracket{\frac{1}{2 \rho} \bp{1 - \frac{\mathcal{M}}{\mathcal{M}_j}}}_{\text{if $U_j$ = $\rho_j$}}}}
\end{multline}









So, we have:

\begin{multline}
    \pd{<\tau_s>}{U_j} = \overbracket{-\bb{\sum_r X_r} \bb{\sum_r \frac{X_r}{\tau_{sr}}}^{-2}}^{-<\tau_s> / \bb{\sum_r X_r / \tau_{sr}}} \times \dots
    \\ \bcb{\sum_r \bb{ \underbracket{\bp{\frac{1}{\sum_i \rho_i \mathcal{M}_i} \bb{\frac{\delta_{jr}}{\mathcal{M}_r} - \frac{X_r}{\mathcal{M}_j}} \frac{1}{\tau_{sr}}}}_{\text{only if $U_j$ = $\rho_j$}} +  \frac{X_r}{\tau_{sr}} \bb{\frac{101325}{p} \pd{p}{U_j} + \frac{1}{3} \pd{T}{U_j} T^{-4/3} A_{sr}}  }}
\end{multline}


Taking this per conserved variable gives:

\begin{align}
    \pd{<\tau_s>}{\rho_j} = -<\tau_s> \frac{1}{\sum_r X_r / \tau_{sr}}
\end{align}


We can rearrange this to:

\begin{multline}
    \pd{<\tau_s>}{U_j} = - \frac{}{} \frac{<\tau_s>}{\sum_r X_r / \tau_{sr}} \bcb{\sum_r \bb{ \bp{\pd{}{U_j} X_r} \frac{1}{\tau_{sr}} + X_r \bp{\pd{}{U_j} \frac{1}{\tau_{sr}}}}}
\end{multline}


\end{flushleft}
\end{document}